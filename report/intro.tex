\section{Introduction}
\label{sec:introduction}

While Global Positioning System (GPS) has been extensively relied on recently for many applications such as navigation, geo-tagging, etc, it doesn't work well for indoor environment due to the block of line-of-sight to satellites. Various approaches have been investigated to achieve precise indoor localization. A sizable number of those researches are trying to localize specific devices or sensor nodes, and many have associate these ``tags'' with human to achieve location-based services. 

Such investigation gets another wave of attention after the emerging smartphones which act naturally as everyone's tag. Though limited by types of sensors, most smartphones are equipped with cellular, Wifi, Bluetooth, inertial measurement unit (IMU), microphone, camera, speaker, etc. Lastest smartphones may even support near field communication (NFC), Bluetooth Low Energy (BLE), proximity sensor, light sensor and even barometer. These sensors have make smartphones not only primarily for communication, but also become such embedded programming platform that many researchers can fast prototype systems and deploy them to a large amount of users.



\cite{JensenEtAl:11:Lab}.


%% Master
%%% Local Variables: 
%%% mode: latex
%%% TeX-master: "ee149"
%%% End: 
