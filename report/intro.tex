\section{Introduction}
\label{sec:introduction}

While Global Positioning System (GPS)~\cite{hofmann1993global} has been extensively relied on recently for many applications - such as navigation and geo-tagging - it doesn't work well for indoor environments due to the blocking of line-of-sight to satellites. Various approaches have been investigated to achieve precise indoor localization, and a sizable number of those researches try to localize specific devices or sensor nodes, and many associate these ``tags'' with humans to achieve location-based services. 

Such investigation gets another wave of attention after the pervasive existence of smartphones which act naturally as everyone's tag. Though limited by types of sensors, most smartphones are equipped with cellular, Wifi, Bluetooth, inertial measurement unit (IMU), microphone, camera, etc. Lastest smartphones may support near field communication (NFC), Bluetooth Low Energy (BLE), proximity sensor, light sensor and even barometer. These sensors have made smartphones not only primarily be used for communication, but also become a embedded programming platform that many researchers can fast prototype and deploy systems to a large amount of users - among them are the indoor localization systems.

Although a fair amount of them are concentrating on providing precise 3D coordinates as the localization results (mainly relying on the RF signal attenuation and triangulation), recently focus has been shifted to address room-level detection as the localization results. This change has been mainly inspired from the essential requirements of most applications, and also has converted the localization problem to a classification problem, where the main challenge is to find the proper signature/feature to assist the classifier.

However, most existing signatures (Cellular~\cite{otsason2005accurate}, Wifi~\cite{cheng2005accuracy}, FM~\cite{chen2012fm}, Acoustic Background Sound~\cite{tarzia2011indoor}), are not precise enough to provide high accuracy of detection. In this project, we propose that it's easy to create reliable man-made signature of each room with considerable amount of infrastructure installation -- by deploying ultrasound beacons in each room. Our consideration of using ultrasound mainly resulted from the observation that the ultrasound signal could be blocked by walls and get reflected inside the room. Another consideration comes from the ability that microphones on smartphones are capable of picking up 20kHZ - 22kHz ultrasound signal. Since we do not require any distance information being calculated, no synchronization or sophiscated encoding scheme is needed. The entire system is fairly simple yet effective.

We will detail our design and implementation in this paper with the structure as follows. Related indoor localization investigation will be discussed in Sec.\,\ref{sec:related-works}. And in Sec.\,\ref{sec:system-architecture}, we focus on the design of the system architecture. The implementation details are in Sec.\,\ref{sec:implementation}, which is followed by the evaluation in Sec.\,\ref{sec:evaluation}. Potential future work lies in Sec.\,\ref{sec:future-work} and Sec.\,\ref{sec:conclusion} concludes this paper.

%% Master
%%% Local Variables: 
%%% mode: latex
%%% TeX-master: "ee149"
%%% End: 
