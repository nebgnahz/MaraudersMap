%% ee149.tex
%% V0.1
%% 2012/11/30
%%
%% http://www.ctan.org/tex-archive/macros/latex/contrib/supported/IEEEtran/
%%

% *** Authors should verify (and, if needed, correct) their LaTeX system  ***
% *** with the testflow diagnostic prior to trusting their LaTeX platform ***
% *** with production work. IEEE's font choices can trigger bugs that do  ***
% *** not appear when using other class files.                            ***
% Testflow can be obtained at:
% http://www.ctan.org/tex-archive/macros/latex/contrib/supported/IEEEtran/testflow

\documentclass[conference, 10pt]{IEEEtran}

% some very useful LaTeX packages include:

\usepackage{cite}
\usepackage{graphicx}
\graphicspath{{./Figures/}}
\DeclareGraphicsExtensions{.png, .pdf}
\usepackage{url}
\usepackage{amsmath}
\usepackage{listings}
\lstset{language=C, basicstyle=\ttfamily\footnotesize, numbers=none, numberstyle=\tiny, tabsize=4, showstringspaces=false}

% correct bad hyphenation here


\begin{document}

% paper title
\title{Marauder's Map -- Achieving Semantic Indoor Localization for Smartphones using Ultrasound}

% author names and affiliations
% use a multiple column layout for up to three different
% affiliations
\author{\IEEEauthorblockN{Ben Zhang}
\IEEEauthorblockA{EECS Department\\
University of California, Berkeley\\
Berkeley, CA 94720\\
E-mail: \url{benzh@eecs.berkeley.edu}}
\and
\IEEEauthorblockN{Hokeun Kim}
\IEEEauthorblockA{EECS Department\\
University of California, Berkeley\\
Berkeley, CA 94720\\
E-mail: \url{hokeunkim@eecs.berkeley.edu}}
\and
\IEEEauthorblockN{Zachary Hargreaves }
\IEEEauthorblockA{EECS Department\\
University of California, Berkeley\\
Berkeley, CA 94704\\
E-mail: \url{hargreaves.z@berkeley.edu}}
}

% make the title area
\maketitle

\begin{abstract}
Indoor localization has been studied for a long time due to a considerable number of applications' requirement. While the ability of providing precise geo-location is important, researches have identified that most applications only require semantic location.

In this paper, we propose the semantic localization concept, and demonstrate its effectiveness. To achieve it, we also designed and implemented an ultrasound-based indoor localization system (named Marauder's Map), which could tell the semantic location of a smartphone holder.
\end{abstract}

%%%%%%%%%%%%%%%%%%%%%%%%%%%%%%%%%%%%%%%%%%
\section{Introduction}
\label{sec:introduction}

While Global Positioning System (GPS)~\cite{hofmann1993global} has been extensively relied on recently for many applications - such as navigation and geo-tagging - it doesn't work well for indoor environments due to the blocking of line-of-sight to satellites. Various approaches have been investigated to achieve precise indoor localization, and a sizable number of those researches try to localize specific devices or sensor nodes, and many associate these ``tags'' with humans to achieve location-based services. 

Such investigation gets another wave of attention after the pervasive existence of smartphones which act naturally as everyone's tag. Though limited by types of sensors, most smartphones are equipped with cellular, Wifi, Bluetooth, inertial measurement unit (IMU), microphone, camera, etc. Lastest smartphones may support near field communication (NFC), Bluetooth Low Energy (BLE), proximity sensor, light sensor and even barometer. These sensors have made smartphones not only primarily be used for communication, but also become a embedded programming platform that many researchers can fast prototype and deploy systems to a large amount of users - among them are the indoor localization systems.

Although a fair amount of them are concentrating on providing precise 3D coordinates as the localization results (mainly relying on the RF signal attenuation and triangulation), recently focus has been shifted to address room-level detection as the localization results. This change has been mainly inspired from the essential requirements of most applications, and also has converted the localization problem to a classification problem, where the main challenge is to find the proper signature/feature to assist the classifier.

However, most existing signatures (Cellular~\cite{otsason2005accurate}, Wifi~\cite{cheng2005accuracy}, FM~\cite{chen2012fm}, Acoustic Background Sound~\cite{tarzia2011indoor}), are not precise enough to provide high accuracy of detection. In this project, we propose that it's easy to create reliable man-made signature of each room with considerable amount of infrastructure installation -- by deploying ultrasound beacons in each room. Our consideration of using ultrasound mainly resulted from the observation that the ultrasound signal could be blocked by walls and get reflected inside the room. Another consideration comes from the ability that microphones on smartphones are capable of picking up 20kHZ - 22kHz ultrasound signal. Since we do not require any distance information being calculated, no synchronization or sophiscated encoding scheme is needed. The entire system is fairly simple yet effective.

We will detail our design and implementation in this paper with the structure as follows. Related indoor localization investigation will be discussed in Sec.\,\ref{sec:related-works}. And in Sec.\,\ref{sec:system-architecture}, we focus on the design of the system architecture. The implementation details are in Sec.\,\ref{sec:implementation}, which is followed by the evaluation in Sec.\,\ref{sec:evaluation}. Potential future work lies in Sec.\,\ref{sec:future-work} and Sec.\,\ref{sec:conclusion} concludes this paper.

%% Master
%%% Local Variables: 
%%% mode: latex
%%% TeX-master: "ee149"
%%% End: 

\section{Related Works}
\label{sec:related-works}


Localization problems have been widely researched because location infromation of people, vehicles or any type of moving targets can be useful for providing the target's behavior, availability, safety and so on. Localizing targets outdoors has become relatively easier thanks to technologies such as global positioning systems (GPS). Indoor localization problems, however, are still under research from various viewpoints, for example, using wireless signals~\cite{liu2007}. Swangmuang and Krishnamurthy~\cite{Swangmuang2008} conducted research on indoor localization technich using fingerprint based positioning with wireless signals. This method used analytical model for received signals of WLANs by approximating probability distribution of error distance to calculate the position of target. There has also been research carried out using an active RFID tag~\cite{Jin2006}.

Semantic indoor localization, which is useful for providing information, not as exact coordinates of a target, but as which part of space the target is located, is what we have undertaken in this project. We divided an environment where we process indoor localization into semantic places such as offices, laboratory, seminar room or sections in a convention center and localize targets and provide this information for users. This kind of semantic place classification was dealt with in Rottmann's work~\cite{rottmann2005semantic}. In this respect, we chose ultrasonic signals, which can be easily blocked by walls or cubicles that divide an environment into semantic spaces, to convey the location information as beacon signals. 



%%% Local Variables: 
%%% mode: latex
%%% TeX-master: "ee149"
%%% End: 


\section{System Architecture}
\label{sec:system-architecture}

We will talk about system architecture in this chapter, and reason about its design.

%% Master
%%% Local Variables: 
%%% mode: latex
%%% TeX-master: "ee149"
%%% End: 

% the folder implementation contains 
% ultrasound, android, backend related tex file
\section{Implementation}
\label{sec:implementation}

\subsection{Ultrasound Beacons}
\label{sec:ultrasound-beacons}

Zachary should write all the implementation of ultrasound in this section.

%% Master
%%% Local Variables: 
%%% mode: latex
%%% TeX-master: "ee149"
%%% End: 

\subsection{Android Application}
\label{sec:android-application}

Hokeun, you can work primarily in this file. use subsubsection if you need to divide it.

%% Master
%%% Local Variables: 
%%% mode: latex
%%% TeX-master: "ee149"
%%% End: 

\subsection{Backend Aggregator}
\label{sec:backend-aggregator}

Ben should talk about the implementation of the backend server and front end html page.

%% Master
%%% Local Variables: 
%%% mode: latex
%%% TeX-master: "ee149"
%%% End: 


%% Master
%%% Local Variables: 
%%% mode: latex
%%% TeX-master: "ee149"
%%% End: 


\section{Evaluation}
\label{sec:evaluation}

In this section, we performed several experiments to characterize such an ultrasound localization system.

\subsection{Power Consumption}
\label{sec:power-consumption}
We have measured the power consumption of our application on LG Revolution VS910 with Android 2.3.4. The summarized measured results are in Table.\,\ref{tab:power}. We could easily found out that having this application running in the background doesn't affect much of the power consumption. While having the application running in front, since there are a lot of screen drawings on the UI update event (either display the spectrum or the map), the application apparently burdens the battery (with an increase of about 59.8\% power consumption than usual).

Since there are so many factors which could potentially influence the power consumption, the measurement is mostly for some rough estimation. But one interesting thing that we do observe by measuring the power consumption is that, most power is used for screen, and more than 100mA current is usually used for radio and communication (there is occasional peaks when the screen is off). 

\begin{table}
  \centering
  \begin{tabular}{|l|c|c|}
    \hline
    & mean (mA) & standard variance (mA) \\
    \hline
    no app, screen on &  214.9 & 43.6 \\
    no app, screen off &  72.4 & 48.0 \\
    \hline
    app runs, screen on &  342.0 & 43.6 \\
    app runs {\em (background)}, screen on & 211.0 & 10.0 \\
    app runs {\em (background)}, screen off &  56.5 & 49.5 \\
    \hline
  \end{tabular}
  \caption{Power consumption of the application}
  \label{tab:power}
\end{table}


% screen on mean: 0.2149 var: 0.0019 -> 0.0436
% screen off mean: 0.0724 var: 0.0023 -> 0.0480
% % Screen on: 0.3256 0.1948 0.3563 0.2035 0.2110 0.1878 0.1874 0.1867 0.2207 0.2261 0.2135 0.2041 0.2014 0.1888 0.1893 0.1888 0.1882 0.1897 0.1913 0.1873 0.2036 0.1968 0.2691 0.2453 lab
% % Dim: 0.1854
% % Screen off: 0.0818 0.0318 0.0099 0.0833 0.0497 0.0062 
% % 0.0051 0.0037 0.0050 0.0041 0.0071 0.1202 0.1035 0.1232 0.0060 0.0052 0.1046 0.0905 0.0317 0.1014 0.0909 0.1588 0.1700 0.0836 0.0965 0.0924 0.1064 0.0568 0.0951 0.0900 0.0766 0.1246 0.0900 0.1034 0.0145 0.0059 0.0550 0.0451 0.0898 0.1050

% Running application
% screen on mean: 0.3420 var: 0.0027 -> 0.0520 
% screen on background mean: 0.2110 var: 0.0001 -> 0.01
% screen off background mean: 0.0565 var: 0.0024 -> 0.0495
% % Screen on: 0.2248 0.2379  0.2325     0.3312 0.3529 0.3512 0.3686 0.3519 0.4118 0.3416 0.3504 0.3339 0.3298 0.3499 0.3162 0.3495 0.3649 0.3471 0.3269 0.3626 0.3561  0.3585 0.3570 0.3495 0.3355 0.3749 0.3629 0.2559 0.4544 0.3623 0.3545 0.4343 0.4395 0.3946 0.3798 0.3525 0.3433 0.3490 0.3696 0.3338 0.3635 0.3433 0.3695 0.3141 0.3861 0.3173 0.3707 0.3736 0.4577 0.3371 0.3577 0.3421 0.3356 0.3393 0.3387 0.3484 0.3566 0.3178 0.3536 0.3573 0.3297 0.3238 0.2341 0.2349 0.2621 0.3493 0.3477 0.3499 0.3412 0.3438 0.3565 0.3502 0.3403 0.3599 0.3514 0.3402 0.3530 0.3416 0.3330 0.3690 0.3484 0.3600 0.2275 0.2181 0.2460 0.2364 0.2379 0.2623 0.3735 0.3502 0.4430 0.4042 0.4276 0.4293 0.3523 0.3373 0.3593 0.3107 0.3767 0.4200 0.3066 0.2086  
% % background screen on
% 0.2136 0.1883 0.2018 0.1941 0.1909 0.1907 0.1880 0.1924 0.1881 0.2056 0.1877 0.2001 0.2554 0.2251 0.2787 0.2756 
% % background screen off
% 0.0769 0.0649 0.0698 0.1014 0.0225 0.0077 0.0047 0.0045 0.0127 0.0043 0.0041 0.0350 0.1009 0.1676 0.0726 0.1196 0.0905 


\subsection{Signal Strength}
\label{sec:signal-strength}
We measure the received siganl strength using commercial laptop (Macbook Pro). The maximum output sound level is 86$\sim$90dB. Since we have converted the received signal to frequency domain, the results are only relative values for the purpose of illustration of ultrasound's characteristics. The obtained figure is shown in Fig.\,\ref{fig:strength}, where we measured four different frequency point, ranging from 20 inches to 140 inches. We conducted our experiment in Room 545K, whose length is around 150 inches. 

This figure shows that generally the received ultrasound signal is attenuating with the increasing of distances. There might exist several points where the received signal strength is larger than those who are nearer; this is mainly caused by all kinds of reflection inside this room. Another observation is that with the increase of signal frequency, the signal strength is descreasing due to the imperfect frequency response of most microphones used in smartphones, but these several signals are still large enough to be differetiated from the nosie.

\begin{figure}
  \centering
  \includegraphics[width=0.9\columnwidth]{result_distance.png}
  \caption{Signal Strength vs. Distance}
  \vspace{-0.3cm}
  \label{fig:strength}
\end{figure}

\subsection{Room Coverage}
\label{sec:room-coverage}

\subsection{Reception Rate}
\label{sec:reception-rate}
In this experiement, we plan to examine the overall detection of our algorithm. Since there are many kinds of noises, reflections around, the data reception rate will not be perfect. Our exponential moving average algorithm (in Sec.\,\ref{sec:android-application}) is used to cope with these imperfections, and we will investigate the effect of this algorithm.

We run our application for 2 mins continously to collect the raw detected results and the filtered results in the normal environment (phone holder moves around during this experiment). The result is shown in Fig.\,\ref{fig:reception}, where we cluster the detected results using the histogram to illustrage the errors. The groundtruth room ID is 176, where the signal is successfully detected most of the time. Bit errors do happen occasionally; from our experiment, the overall reception rate is 67.60\%. After the filtering, some sporadic errors are corrected, and the overall detection has been improved to 87.50\%. 

Further observation reveals that most of the wrong detection resulted in Room ID 0, which means no reliable signal at all. We could just infer users' location as previous room since it's not physically feasible of changing room drastically; such semantic combination with real scenario will yield 97.25\% accuracy.

\begin{figure}
  \centering
  \includegraphics[width=0.9\columnwidth]{ReceptionRate}
  \caption{Histogram of the raw detected room ID ({\em up}); histogram of filtered result ({\em down})}
  \vspace{-0.3cm}
  \label{fig:reception}
\end{figure}


\subsection{Obstacle Blocking}
\label{sec:obstacle-blocking}
As we have argued in Sec.\,\ref{sec:introduction} that the ultrasound can be easily reflected inside a room while the wall can be perfect deliminator of spaces. We conducted experiment of validating such idea by measuring both the signal strength and the detection rate. We place the phone on both side of the glass wall of Room 545K, and run this application both for more than 1 minute. The measurement after FFT is shown in Fig.\,\ref{fig:wallblock}. And if we calculate the detected results, the inside experiment could correctly detect the room ID, with an accuracy of 96.06\% (it's 91.25\% before the filtering) while the outside experiment always outputs 0.

\begin{figure}
  \centering
  \includegraphics[width=0.9\columnwidth]{WallBlocking}
  \vspace{-0.3cm}
  \caption{The effect of wall blocking}
  \label{fig:wallblock}
\end{figure}

\subsection{Noise Resistance}
\label{sec:noise-resistance}



%% Master
%%% Local Variables: 
%%% mode: latex
%%% TeX-master: "ee149"
%%% End: 


\section{Future Work}
\label{sec:future-work}

In this project, we've investigated the possibility of utilizing ultrasound to achieve semantic localization. The characteristics of ultrasound is nice to cover each room and not crossing any walls. This is perfect for room-level detection and potentially for cubicle-level (since there is still such blocking board between cubicles). But for open area such as the exploratorium at San Fransisco, this won't work such well. But the inaudibility of ultrasound and the slow speed (in comparison to light and electromagnetic wave) will enable time-of-flight (TOF) based localization. Further investigation can be done for this extension.

On the other hand, we haven't utilized other sensors on the mobile phone to perform dead reckoning (DR). Many recent researches have been showing that using particle filters for the accelerometer readings can determine the trajectory of users. But such approach is undermined by the drift caused by the error of sensor measurement. The ultrasound localization could act as landmarks to correct such drift, and many works might be done in this direction.
%% Master
%%% Local Variables: 
%%% mode: latex
%%% TeX-master: "ee149"
%%% End: 
\section{Conclusion}
\label{sec:conclusion}

The main goal of this project was to implement a feasible semantic indoor localization system that can be used to locate a user's position on a map who is carrying a smartphone so that the person does not need any additional device. Producing information of the semantic spaces was realized by using an ultrasound signal generator that generates sinusoids with specific frequencies and the sinusoids conveyed IDs of rooms in the environment. An Android application was implemented in order to collect audio samples to be decoded into room information using Fast Fourier Transform (FFT) and a couple of filtering algorithms. Calculated location information could be either sent to a server that can aggregate multiple targets' location or kept secret for the user's privacy. This main goal of the project was mostly achieved throughout this semester and we illustrated how accurate and efficient this localization system is by various kinds of experiments. Power consumption caused by this Android application imposed mild overhead on the smartphone, this system showed clear and accurate results even though when there was noise and interference. Through additional evaluation such as tests for room coverage and obstacle blocking, we also proved this system is proper for indoor characteristics. 

%% Master
%%% Local Variables: 
%%% mode: latex
%%% TeX-master: "ee149"
%%% End: 


%% template to put figure
% Layers figure
% \begin{figure}[t]
%   \includegraphics[width=\linewidth]{figFileName}
%   \caption{This is the caption of this figure.}
%   \label{fig:label}
% \end{figure}

%%%%%%%%%%%%%%%%%%%%%%%%%%%%%%%%%%%%%%%%%%

% use section* for acknowledgement
\section*{Acknowledgment}
To begin with, we would like to express our gratitude to Professor Edward A. Lee and Professor Sanjit A. Seshia for teaching and guiding us to finish this course and course project successfully. We also thank our GSI, Zach Wasson for helping us to learn practical knowledge through labs and giving us feedbacks, and our project mentor, Tomi R{\"a}ty for giving us advice on the direction of this project and frequent discussion on the progress. And we also want to thank all the other people with whom we had interesting discussions about the indoor localization problem.

% references section
% NOTE: BibTeX documentation can be easily obtained at:
% http://www.ctan.org/tex-archive/biblio/bibtex/contrib/doc/

% can use a bibliography generated by BibTeX as a .bbl file
% standard IEEE bibliography style from:
% http://www.ctan.org/tex-archive/macros/latex/contrib/supported/IEEEtran/bibtex
\bibliographystyle{IEEEtran}
% argument is your BibTeX string definitions and bibliography database(s)
%\bibliography{IEEEabrv,../bib/paper}
%
% <OR> manually copy in the resultant .bbl file
% set second argument of \begin to the number of references
% (used to reserve space for the reference number labels box)
%\bibliographystyle{jponewurl}
\bibliography{Refs}
%%%%%%%%%%%%%%%%%%%%%%%%%%%%%%%%%%%%%%%%%%%%%%%%%%%%%%%%%%%%%%%%
%%%%%% NOTE: I didn't delete the origin entries in Refs.bib %%%%
%%%%%% to give us a flavor of format of bibtex (Ben)        %%%%
%%%%%%%%%%%%%%%%%%%%%%%%%%%%%%%%%%%%%%%%%%%%%%%%%%%%%%%%%%%%%%%%

\appendix

\subsection{Role Played in the Project}
\label{sec:role-played-project}
% each member's role in one paragraph
Ben Zhang working on Backend server.
Hokeun Kim has mostly worked on developing the Android application, implementing Fast Fourier Transform (FFT) on collected audio samples, decoding of location information from the FFT results on ultrasonic signals and two-level filtering algorithms to stabilize decoded results. He also designed the user interface of Android application.
Zachary Hargreaves primarily developed the hardware for the ultrasound beacon.  This task required constuction of code to generate a digital sine wave then processing of the signal in order to be compatable with the ultra sound reciever.   

\subsection{Related with EE149 Course}
\label{sec:related-with-ee149}
\subsubsection{Key concepts learned from class}
\label{sec:key-concepts-learned}

Interfacing with sensors (microphone) and understanding the imperfect nature of sensor signals proved valuable when filtering high frequency disturbances in our signals.  The filtering methods introduced in the first lab were directly related to the processing required in our project.  Learnig low-level uC programming was useful when decoding the data sheets in order to understand the role each register played in implementing the ISR.  System modeling helped us avoid future probelems before implementing our network protocol.

\subsubsection{Course feedback}
\label{sec:course-feedback}
Pretty good course. Having more hardware components around the lab would help many projects out. This would streamline the debugging process and encourage cross group collaboration because many groups using similar (class provided) components would be able to help one another out.  

\subsubsection{Technical Challenges}
\label{sec:technical-challenges}
Aside from the bottle necks caused by delivery times for specialized components, understanding amplification circuits for low powered systems was a large technical challenge.
For decoding location information from ultrasounds, coping with noise and interference caused by environments and signal itself was the biggest technical challenge. To deal with varying signal strength according to frequency range and distance from the ultrasound generator was another challenge for developing the Android application.
TODO
 - Noise and interference for the signal   
- stable server 

%% Master
%%% Local Variables: 
%%% mode: latex
%%% TeX-master: "ee149"
%%% End:


\end{document}