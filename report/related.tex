\section{Related Works}
\label{sec:related-works}


Localization problems have been widely researched because location infromation of people, vehicles or any type of moving targets can be useful for providing the target's behavior, availability, safety and so on. Localizing targets outdoors has become relatively easier thanks to technologies such as global positioning systems (GPS). Indoor localization problems, however, are still under research from various viewpoints, for example, using wireless signals~\cite{liu2007}. Swangmuang and Krishnamurthy~\cite{Swangmuang2008} conducted research on indoor localization technich using fingerprint based positioning with wireless signals. This method used analytical model for received signals of WLANs by approximating probability distribution of error distance to calculate the position of target. There has also been research carried out using an active RFID tag~\cite{Jin2006}.

Semantic indoor localization, which is useful for providing information, not as exact coordinates of a target, but as which part of space the target is located, is what we have undertaken in this project. We divided an environment where we process indoor localization into semantic places such as offices, laboratory, seminar room or sections in a convention center and localize targets and provide this information for users. This kind of semantic place classification was dealt with in Rottmann's work~\cite{Rottmann2005}. In this respect, we chose ultrasonic signals, which can be easily blocked by walls or cubicles that divide an environment into semantic spaces, to convey the location information as beacon signals. 



%%% Local Variables: 
%%% mode: latex
%%% TeX-master: "ee149"
%%% End: 
