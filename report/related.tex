\section{Related Works}
\label{sec:related-works}

Localization problems have been widely researched because location information of people, vehicles or any type of moving targets can be useful for providing the target's behavior, availability, safety and so on. Localizing targets outdoors has become relatively easier thanks to the technologies such as (GPS). Indoor localization problems, however, are still under research from various viewpoints. Early investigations include the Active Badge~\cite{want1992active} system uses wearable infrared transmitters. And there were many approaches using wireless signals to achieve indoor localization (see the survey paper in \cite{liu2007}). These wireless signals include cell phone signals such as cellular~\cite{otsason2005accurate} that are most common in daily life nowadays, which have advantages in the sense that this method does not demand any additional equipment for localization. Swangmuang and Krishnamurthy~\cite{Swangmuang2008} conducted research on indoor localization technique using fingerprint based positioning with wireless signals. FM~\cite{chen2012fm} signals, with its pervasiveness, could also be leveraged to achieve localization. There has also been research carried out using an active RFID tag~\cite{Jin2006, buettner2009recognizing}.

Semantic indoor localization, which is useful for providing information, not as exact coordinates of a target, but as which part of space the target is located, is what we have undertaken in this project. We divided an environment where we process indoor localization into semantic places such as offices, laboratories, seminar rooms or sections in a convention center and localize targets and provide this information for users. This kind of semantic place classification was dealt with in Rottmann's work~\cite{rottmann2005semantic}. In this respect, we chose ultrasonic signals, which can be easily blocked by walls or cubicles that divide an environment into semantic spaces, to convey the location information as beacon signals. There has been research using ultrasound signals for indoor localization with ultrasonic sensors~\cite{feng1997mobile, priyantha2005cricket}, which is closely related to our project, although they used ultrasounds to measure distance with the method of triangulation rather than carry information as we did in this project.



%%% Local Variables: 
%%% mode: latex
%%% TeX-master: "ee149"
%%% End: 
