\section{Evaluation}
\label{sec:evaluation}

In this section, we performed several experiments to characterize such an ultrasound localization system.

\subsection{Power Consumption}
\label{sec:power-consumption}
We have measured the power consumption of our application on LG Revolution VS910 with Android 2.3.4. The summarized measured results are in Table.\,\ref{tab:power}. We could easily found out that having this application running in the background doesn't affect much of the power consumption. While having the application running in front, since there are a lot of screen drawings on the UI update event (either display the spectrum or the map), the application apparently burdens the battery (with an increase of about 59.8\% power consumption than usual).

Since there are so many factors which could potentially influence the power consumption, the measurement is mostly for some rough estimation. But one interesting thing that we do observe by measuring the power consumption is that, most power is used for screen, and more than 100mA current is usually used for radio and communication (there is occasional peaks when the screen is off). 

\begin{table}
  \centering
  \begin{tabular}{|l|c|c|}
    \hline
    & mean (mA) & standard variance (mA) \\
    \hline
    no app, screen on &  214.9 & 43.6 \\
    no app, screen off &  72.4 & 48.0 \\
    \hline
    app runs, screen on &  342.0 & 43.6 \\
    app runs {\em (background)}, screen on & 211.0 & 10.0 \\
    app runs {\em (background)}, screen off &  56.5 & 49.5 \\
    \hline
  \end{tabular}
  \caption{Power consumption of the application}
  \label{tab:power}
\end{table}


% screen on mean: 0.2149 var: 0.0019 -> 0.0436
% screen off mean: 0.0724 var: 0.0023 -> 0.0480
% % Screen on: 0.3256 0.1948 0.3563 0.2035 0.2110 0.1878 0.1874 0.1867 0.2207 0.2261 0.2135 0.2041 0.2014 0.1888 0.1893 0.1888 0.1882 0.1897 0.1913 0.1873 0.2036 0.1968 0.2691 0.2453 lab
% % Dim: 0.1854
% % Screen off: 0.0818 0.0318 0.0099 0.0833 0.0497 0.0062 
% % 0.0051 0.0037 0.0050 0.0041 0.0071 0.1202 0.1035 0.1232 0.0060 0.0052 0.1046 0.0905 0.0317 0.1014 0.0909 0.1588 0.1700 0.0836 0.0965 0.0924 0.1064 0.0568 0.0951 0.0900 0.0766 0.1246 0.0900 0.1034 0.0145 0.0059 0.0550 0.0451 0.0898 0.1050

% Running application
% screen on mean: 0.3420 var: 0.0027 -> 0.0520 
% screen on background mean: 0.2110 var: 0.0001 -> 0.01
% screen off background mean: 0.0565 var: 0.0024 -> 0.0495
% % Screen on: 0.2248 0.2379  0.2325     0.3312 0.3529 0.3512 0.3686 0.3519 0.4118 0.3416 0.3504 0.3339 0.3298 0.3499 0.3162 0.3495 0.3649 0.3471 0.3269 0.3626 0.3561  0.3585 0.3570 0.3495 0.3355 0.3749 0.3629 0.2559 0.4544 0.3623 0.3545 0.4343 0.4395 0.3946 0.3798 0.3525 0.3433 0.3490 0.3696 0.3338 0.3635 0.3433 0.3695 0.3141 0.3861 0.3173 0.3707 0.3736 0.4577 0.3371 0.3577 0.3421 0.3356 0.3393 0.3387 0.3484 0.3566 0.3178 0.3536 0.3573 0.3297 0.3238 0.2341 0.2349 0.2621 0.3493 0.3477 0.3499 0.3412 0.3438 0.3565 0.3502 0.3403 0.3599 0.3514 0.3402 0.3530 0.3416 0.3330 0.3690 0.3484 0.3600 0.2275 0.2181 0.2460 0.2364 0.2379 0.2623 0.3735 0.3502 0.4430 0.4042 0.4276 0.4293 0.3523 0.3373 0.3593 0.3107 0.3767 0.4200 0.3066 0.2086  
% % background screen on
% 0.2136 0.1883 0.2018 0.1941 0.1909 0.1907 0.1880 0.1924 0.1881 0.2056 0.1877 0.2001 0.2554 0.2251 0.2787 0.2756 
% % background screen off
% 0.0769 0.0649 0.0698 0.1014 0.0225 0.0077 0.0047 0.0045 0.0127 0.0043 0.0041 0.0350 0.1009 0.1676 0.0726 0.1196 0.0905 


\subsection{Signal Strength}
\label{sec:signal-strength}
We measure the received siganl strength using commercial laptop (Macbook Pro). The maximum output sound level is 86$\sim$90dB. Since we have converted the received signal to frequency domain, the results are only relative values for the purpose of illustration of ultrasound's characteristics. The obtained figure is shown in Fig.\,\ref{fig:strength}, where we measured four different frequency point, ranging from 20 inches to 140 inches. We conducted our experiment in Room 545K, whose length is around 150 inches. 

This figure shows that generally the received ultrasound signal is attenuating with the increasing of distances. There might exist several points where the received signal strength is larger than those who are nearer; this is mainly caused by all kinds of reflection inside this room. Another observation is that with the increase of signal frequency, the signal strength is descreasing due to the imperfect frequency response of most microphones used in smartphones, but these several signals are still large enough to be differetiated from the nosie.

\begin{figure}
  \centering
  \includegraphics[width=0.9\columnwidth]{result_distance.png}
  \caption{Signal Strength vs. Distance}
  \vspace{-0.3cm}
  \label{fig:strength}
\end{figure}

\subsection{Room Coverage}
\label{sec:room-coverage}

\subsection{Reception Rate}
\label{sec:reception-rate}
In this experiement, we plan to examine the overall detection of our algorithm. Since there are many kinds of noises, reflections around, the data reception rate will not be perfect. Our exponential moving average algorithm (in Sec.\,\ref{sec:android-application}) is used to cope with these imperfections, and we will investigate the effect of this algorithm.

We run our application for 2 mins continously to collect the raw detected results and the filtered results in the normal environment (phone holder moves around during this experiment). The result is shown in Fig.\,\ref{fig:reception}, where we cluster the detected results using the histogram to illustrage the errors. The groundtruth room ID is 176, where the signal is successfully detected most of the time. Bit errors do happen occasionally; from our experiment, the overall reception rate is 67.60\%. After the filtering, some sporadic errors are corrected, and the overall detection has been improved to 87.50\%. 

Further observation reveals that most of the wrong detection resulted in Room ID 0, which means no reliable signal at all. We could just infer users' location as previous room since it's not physically feasible of changing room drastically; such semantic combination with real scenario will yield 97.25\% accuracy.

\begin{figure}
  \centering
  \includegraphics[width=0.9\columnwidth]{ReceptionRate}
  \caption{Histogram of the raw detected room ID ({\em up}); histogram of filtered result ({\em down})}
  \vspace{-0.3cm}
  \label{fig:reception}
\end{figure}


\subsection{Obstacle Blocking}
\label{sec:obstacle-blocking}
As we have argued in Sec.\,\ref{sec:introduction} that the ultrasound can be easily reflected inside a room while the wall can be perfect deliminator of spaces. We conducted experiment of validating such idea by measuring both the signal strength and the detection rate. We place the phone on both side of the glass wall of Room 545K, and run this application both for more than 1 minute. The measurement after FFT is shown in Fig.\,\ref{fig:wallblock}. And if we calculate the detected results, the inside experiment could correctly detect the room ID, with an accuracy of 96.06\% (it's 91.25\% before the filtering) while the outside experiment always outputs 0.

\begin{figure}
  \centering
  \includegraphics[width=0.9\columnwidth]{WallBlocking}
  \vspace{-0.3cm}
  \caption{The effect of wall blocking}
  \label{fig:wallblock}
\end{figure}

\subsection{Noise Resistance}
\label{sec:noise-resistance}



%% Master
%%% Local Variables: 
%%% mode: latex
%%% TeX-master: "ee149"
%%% End: 
